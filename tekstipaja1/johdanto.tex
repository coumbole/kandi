\documentclass[12pt]{article}
\usepackage[paper=a4paper,margin=1in]{geometry}
\usepackage{lmodern}
\usepackage{amssymb,amsmath}
\usepackage{ifxetex,ifluatex}
\usepackage{fixltx2e} % provides \textsubscript
\ifnum0\ifxetex1\fi\ifluatex1\fi=0 % if pdftex
  \usepackage[T1]{fontenc}
  \usepackage[utf8]{inputenc}
\else % if luatex or xelatex
  \ifxetex\usepackage{mathspec}
  \else
    \usepackage{fontspec}
  \fi
  \defaultfontfeatures{Ligatures=TeX,Scale=MatchLowercase}
\fi
% use upquote if available, for straight quotes in verbatim environments
\IfFileExists{upquote.sty}{\usepackage{upquote}}{}
% use microtype if available
\IfFileExists{microtype.sty}{%
\usepackage[]{microtype}
\UseMicrotypeSet[protrusion]{basicmath} % disable protrusion for tt fonts
}{}
\PassOptionsToPackage{hyphens}{url} % url is loaded by hyperref
\usepackage[unicode=true]{hyperref}
\urlstyle{same}  % don't use monospace font for urls
\IfFileExists{parskip.sty}{%
\usepackage{parskip}
}{% else
\setlength{\parindent}{0pt}
\setlength{\parskip}{6pt plus 2pt minus 1pt}
}
\setlength{\emergencystretch}{3em}  % prevent overfull lines
\providecommand{\tightlist}{%
  \setlength{\itemsep}{0pt}\setlength{\parskip}{0pt}}
\setcounter{secnumdepth}{0}
% Redefines (sub)paragraphs to behave more like sections
\ifx\paragraph\undefined\else
\let\oldparagraph\paragraph\renewcommand{\paragraph}[1]{\oldparagraph{#1}\mbox{}}
\fi
\ifx\subparagraph\undefined\else
\let\oldsubparagraph\subparagraph\renewcommand{\subparagraph}[1]{\oldsubparagraph{#1}\mbox{}}
\fi


\usepackage{fancyhdr}

\pagestyle{fancy} 
\fancyhf{}

\renewcommand{\headrulewidth}{0pt}
\fancyhead[R]{Ville Kumpulainen \\ 529387}
\setlength{\headheight}{2\baselineskip}

% set default figure placement to htbp
\makeatletter
\def\fps@figure{htbp}
\makeatother


\renewcommand{\baselinestretch}{1.25}
\date{}

\begin{document}


\section*{Yhteenveto}


\subsection*{Adopting agile software development in large organizations}

Ketterät ohjelmistokehitysmetodit ovat kasvattaneet suosiotaan
tasaisesti sitten niiden lanseeraamisen vuonna 2001. Ketterät menetelmät
on suunniteltu alun perin pienille, noin 10 hengen ryhmille, mutta
niiden edut ovat tehneet niistä houkuttelevan vaihtoehdon myös suurille
organisaatioille. Siirtymä perinteisestä vesiputousmallista
ohjelmistotuotantoprosessina ketteriin menetelmiin on merkittävä ja
haasteellinen. Monet suuret organisaatiot ovat viimeisen vuosikymmenen
aikana yrittäneet toteuttaa tätä siirtymää. Jotkut näistä ovat
onnistuneet, mutta merkittävä osa transformaatioista edelleen
epäonnistuu. Luotettavasti toimivaa suurille organisaatioille
suunniteltua ketterää metodologiaa ei ole onnistuttu kehittämään, vaikka
aihetta on tutkittu paljon.

Tämän tutkimuksen tarkoituksena on selvittää, miksi suuret organisaatiot
aloittavat siirtymän ja miten ne etenevät. Ensimmäisen
tutkimuskysymyksen avulla pyritään selvittämään muutospäätöksen
taustalla olevia syitä. Tämän oletetaan paljastavan vesiputousmallin
ongelmia ja auttavan hahmottamaan, minkälaisia parannuksia ketteriltä
menetelmiltä odotetaan. Toinen kysymys tutkii siirtymien etenemistä.
Tämän avulla on tarkoitus selvittää, minkälaiset tekijät johtavat
onnistuneisiin siirtymiin ja mitkä valinnat sitä sen aikana. Näin
luodaan pohjaa myöhemmälle tutkimukselle, joka mahdollistaa yleisesti
toimivan siirtymämallin luomisen.

Tässä tutkimuksessa suurella orgnaisaatiolla tarkoitetaan vähintään
50:tä henkeä tai 6:ta erillista työryhmää kattavaa
ohjelmistokehitysorganisaatiota. Tämä organisaatio voi sisältää
varsinaisten ohjelmistokehittäjien lisäksi myös ohjelmistoarkkitehtejä
ja projektipäälliköitä.

Tutkimusmenetelmäni oli systemaattinen kirjallisuuskatsaus. Suoritin
tutkimuksen alkuvaiheessa avainsanapohjaisia hakuja neljään eri
tietokantaan, minka seurauksena sain aineistokseni 1182 eri tutkimusta.
Lähdin karsimaan näitä artikkeleita ensin tiivistelmien perusteella ja
myöhemmin koko tekstien perusteella. Karsimisessa oli olennaista, että
tutkimus täyttää tietyt ennaltamäärätyt valintakriteerit. Kriteerit
edellyttivät, että tutkimus käsittelee ketterää ohjelmistokehitystä,
organisaation laajuista transformaatiota, organisaatio on riittävän
suuri, ja että tutkimus on empiirinen ja sisältää raportointia jostakin
konkreettisesta toteutuneesta siirtymästä. Tiivistelmien perusteella
suoritetusta karsimisesta jäi jäljelle 43 artikkelia. Näistä karsin koko
tekstien perusteella ylimääräiset pois, jolloin jäljelle jäi 6
tutkimusta, jotka toteuttivat kaikki valitsemiskriteerit.

Karsimisen jälkeen nämä kuusi niin kutsuttua primäärilähdettä koodattiin
niiden sisältämien asiasanojen perusteella. Koodit liittyivät siirtymien
syihin, itse siirtymäprosessiin, yritysten käyttöönottamiin uusiin
käytäntöihin ja siirtymiseen liittyviin sijoituksiin. Lisäksi yksi
muista hieman erillinen koodiperhe liittyi artikkelien kontekstiin, eli
esimerkiksi maantieteelliseen sijaintiin ja organisaation kokoon.

Tutkiessani ensin ketteriin menetelmiin siirtymiseen johtaneita syitä,
löysin kolme pääkategoriaa, joihin kaikki yksittäiset raportoidut syyt
kuuluivat. Ensimmäinen kategoria liittyi yritysten liiketoimintaan,
toinen prosesseihin ja kolmas projektien johtamiseen.

Tärkein liiketoimintaan vaikuttava ongelma oli vanhan prosessin
jäykkyys. Prosessi eikä kyennyt vastaamaan tehokkaasti asiakkaiden
muuttuviin tarpeisiin ja vaatimuksiin. Toinen ongelma oli pitkä
toimitusaika. Projektien kesto oli usein jopa useita vuosia, minkä
aikana sekä asiakkaan liiketoimintaympäristö että markkinat ehtivät
muuttua merkittävästi. Tässä ajassa rakennettu ohjelmistotuote ehti
pahimmillaan vanhentua ennen julkaisuaan. Kolmas syy oli kilpailukyvyn
ylläpitäminen. Nuoremmat ja pienikokoisemmat yritykset pystyivät
kilpailemaan samoista asiakkaista toimimalla joustavammin ja
ketterämmin. Lisäksi ne pystyivät toimittamaan ohjelmistotuotteita
asiakkaille nopeammin ketterien mentelmien avulla.

Toinen syykategoria oli ohjelmistotuotantoprosessiin liittyvät ongelmat.
Vanhaa prosessia yritettiin parantaa uusilla versioilla ja räätälöimällä
sitä uusilla säännöillä. Lopulta tämä jatkuva sääntöjen kehittely loi
kuitenkin nk. ``sääntöviidakon'', joka teki prosessin
jatkokehittämisestä käytännössä mahdotonta. Näistä pyrittiin pääsemään
eroon siirtymällä ketteriin menetelmiin, jotka suosivat ihmisiä ja
heidän välistä kommunikaatiota jäykkien prosessien, mallien ja
työkalujen sijaan.

Kolmas syykategoria liittyi projektin johtamiseen. Vanhassa
organisaatiomallissa yrityksen eri osastot toimivat toisistaan
eristyksissä niin kutsutuissa siiloissa. Näiden välinen viestintä oli
lähes olematonta, mikä teki projektien toimittamisesta tehotonta. Eri
osastot olivat eri aikoihin valmiita tiettyjen projektin vaiheiden
kanssa, jolloin toiset osastot joutuivat odottelemaan näiden
valmistumista, ennen kuin pääsivät omien töidensä kanssa eteenpäin.
Ketterät menetelmät vastaavat tähän ongelmaan uudenlaisella
organisaatiorakenteella. Tarkoituksena on rakentaa yrityksen sisäisiä
työryhmiä nimenomaan tiettyjen toiminnallisuuksien perusteella, eikä
virkanimikkeiden mukaan.

Toinen tutkimuskysymys pyrki selvittämään, miten
ohjelmistotuotantoprosessin muuttaminen etenee suurissa
organisaatioissa. Lähes kaikissa primäärilähteissä muutos lähti ylimmän
johdon aloitteesta. Liiketoimintaan liittyvät ongelmat havaittiin heidän
toimestaan, ja ketteriä menetelmiä pidettiin yleisesti potentiaalisena
ratkaisuna. Valtaosassa tutkittuja yrityksiä muutos oli kuitenkin hidas
ja vaiheittainen. Ensin toteutettiin taustatutkimuksia, joilla pyrittiin
selvittämään työntekijöiden tämänhetkinen tietotaso ketteristä
menetelmistä, yrityksen suurimmat ongelmat nykyisen prosesin kanssa,
sekä mihin ketteriin menetelmiin yrityksen kannattaisi siirtyä.
Tutkimuksen ja päätöksen teon jälkeen uusia menetelmiä lähdettiin ensin
kokeilemaan pilottiprojektien kautta. Pilotit kestivät puolesta vuodesta
vuoteen, jonka jälkeen uskallettiin siirtyä ketteriin menetelmiin koko
yrityksen tasolla.

Transformaation aikana lähes kaikissa yrityksissä otettiin käyttöön
samoja käytäntöjä. Ensinnäkin yritykset hyödynsivät ahkerasti ketteriin
menetelmiin erikoistuneita ulkoisia konsultteja, joilla oli kokemusta
vastaavien muutosten toteuttamisesta muissa suurissa organisaatioissa.
Tämä koettiin tarpeelliseksi, koska suurin epäonnistumisen uhka liittyy
nimenomaan tietotaidon puutteeseen ja siihen, että työntekijät palaavat
vaistomaisesti takaisin vanhoihin tapoihin. Toisekseen valtaosa
yrityksistä otti käyttöön erilaisia tiedonjakoalustoja. Näiden
tarkoituksena oli vastata osastojen siiloutumiseen liittyvään ongelmaan
ja lisätä toiminnan läpinäkyvyyttä koko yrityksen tasolla. Kolmas
käytäntö oli jatkuva integraatio. Tämä tarkoittaa sitä, että jokaisena
päivänä tehdyt pienet muutokset ohjelmistotuotteeseen viedään
välittömästi tuotantoon ja asiakkaan testattavaksi. Tämä johtaa
välittömään palautteeseen, jolloin tuotetta voidaan kehittää
yhteistyössä asiakkaan kanssa koko kehitystyön ajan.

Johtopäätöksenä voidaan siis todeta, että vanhasta vesiputousmallisesta
ohjelmistokehityksestä siirrytään pois, koska nykyaikana asiakkaiden ja
markkinoiden toiveet ja tarpeet muuttuvat nopeammin, kuin mihin vanha
prosessi pystyi vastaamaan. Siirtyminen ketteriin menetelmiin
toteutettiin pääsääntöisesti ylimmän johdon aloitteesta hyvin
suunnitellusti ja hallitusti. Lisäksi siirtymän yhteydessä otettiin
käyttöön uusia toimintatapoja, jotka tehostivat yritysten toimintaa ja
yhteistyötä sekä sisäisesti että asiakkaan kanssa.

\end{document}
