\documentclass[12pt]{article}
\usepackage[paper=a4paper,margin=0.9in]{geometry}
\usepackage{lmodern}
\usepackage{amssymb,amsmath}
\usepackage{ifxetex,ifluatex}
\usepackage{fixltx2e} % provides \textsubscript
\ifnum0\ifxetex1\fi\ifluatex1\fi=0 % if pdftex
  \usepackage[T1]{fontenc}
  \usepackage[utf8]{inputenc}
\else % if luatex or xelatex
  \ifxetex\usepackage{mathspec}
  \else
    \usepackage{fontspec}
  \fi
  \defaultfontfeatures{Ligatures=TeX,Scale=MatchLowercase}
\fi
% use upquote if available, for straight quotes in verbatim environments
\IfFileExists{upquote.sty}{\usepackage{upquote}}{}
% use microtype if available
\IfFileExists{microtype.sty}{%
\usepackage[]{microtype}
\UseMicrotypeSet[protrusion]{basicmath} % disable protrusion for tt fonts
}{}
\PassOptionsToPackage{hyphens}{url} % url is loaded by hyperref
\usepackage[unicode=true]{hyperref}
\urlstyle{same}  % don't use monospace font for urls
\IfFileExists{parskip.sty}{%
\usepackage{parskip}
}{% else
\setlength{\parindent}{0pt}
\setlength{\parskip}{6pt plus 2pt minus 1pt}
}
\setlength{\emergencystretch}{3em}  % prevent overfull lines
\providecommand{\tightlist}{%
  \setlength{\itemsep}{0pt}\setlength{\parskip}{0pt}}
\setcounter{secnumdepth}{0}
% Redefines (sub)paragraphs to behave more like sections
\ifx\paragraph\undefined\else
\let\oldparagraph\paragraph\renewcommand{\paragraph}[1]{\oldparagraph{#1}\mbox{}}
\fi
\ifx\subparagraph\undefined\else
\let\oldsubparagraph\subparagraph\renewcommand{\subparagraph}[1]{\oldsubparagraph{#1}\mbox{}}
\fi


\usepackage{fancyhdr}

\pagestyle{fancy} 
\fancyhf{}

\renewcommand{\headrulewidth}{0pt}
\fancyhead[R]{Ville Kumpulainen \\ 529387}
\setlength{\headheight}{2\baselineskip}

% set default figure placement to htbp
\makeatletter
\def\fps@figure{htbp}
\makeatother


\renewcommand{\baselinestretch}{1.5}
\date{}

\begin{document}


\section*{Yhteenveto}


\subsection*{Adopting agile software development in large organizations}


Ketterät ohjelmistokehitysmetodit ovat kasvattaneet suosiotaan
tasaisesti sitten niiden lanseeraamisen vuonna 2001. Ketterät
menetelmät on alun perin suunniteltu pienille, noin 10
hengen ryhmille. Niiden edut ovat kuitenkin tehneet niistä
houkuttelevan vaihtoehdon myös suurille organisaatioille. Siirtymä
ohjelmistotuotantoprosesseissa perinteisestä vesiputousmallista
ketteriin menetelmiin on merkittävä ja haasteellinen. Monet suuret
organisaatiot ovat viimeisen vuosikymmenen aikana yrittäneet toteuttaa
tätä siirtymää. Jotkut näistä ovat onnistuneet, mutta merkittävä
osa transformaatioista edelleen epäonnistuu. Luotettavasti toimivaa
suurille organisaatioille suunniteltua ketterää siirtymämallia ei ole
onnistuttu kehittämään, vaikka aihetta on tutkittu paljon.

Tämän tutkimuksen tarkoituksena on selvittää, miksi suuret
organisaatiot aloittavat siirtymän ja miten nämä siirtymät
etenevät. Ensin selvitän muutospäätöksen taustalla olevia syitä,
mikä paljastaa vesiputousmallin ongelmia sekä auttaa
hahmottamaan minkälaisia parannuksia ketteriltä menetelmiltä
odotetaan. Toiseksi tutkin siirtymien etenemistä eli sitä,
minkälaiset tekijät ja valinnat johtavat onnistuneisiin siirtymiin.
Näin pyrin luomaan pohjaa myöhemmälle tutkimukselle, joka
mahdollistaa yleisesti toimivan siirtymämallin luomisen.

Tässä tutkimuksessa suurella organisaatiolla tarkoitetaan sellaista
ohjelmistokehitysorganisaatiota, joka kattaa vähintään 50
henkeä tai 6 erillistä työryhmää. Tällainen organisaatio voi
sisältää varsinaisten ohjelmistokehittäjien lisäksi myös
ohjelmistoarkkitehtejä ja projektipäälliköitä.

Tutkimusmenetelmäni oli systemaattinen kirjallisuuskatsaus. Suoritin
tutkimuksen alkuvaiheessa avainsanapohjaisia hakuja neljään eri
tietokantaan, minka seurauksena sain aineistokseni 1182 eri tutkimusta.
Karsin näitä artikkeleita ensin tiivistelmien ja sittemmin
koko tekstien perusteella sen mukaan, täyttivätkö ne tietyt
ennaltamäärätyt valintakriteerit. Kriteereiden mukaan tutkimuksen
tuli käsitellä ketterää ohjelmistokehitystä ja koko organisaation
laajuista transformaatiota. Lisäksi tutkitun organisaation tuli olla
riittävän suuri. Rajasin mukaan otettavat artikkelit sellaisiin, jotka
oli toteutettu empiirisesti ja jotka sisälsivät raportointia jostakin
konkreettisesta toteutuneesta siirtymästä. Karsimisesta jäi jäljelle
6 kaikki valitsemiskriteerit täyttävää tutkimusta.

Karsimisen jälkeen koodasin nämä kuusi niin kutsuttua
primäärilähdettä niiden sisältämien asiasanojen perusteella.
Koodit liittyivät siirtymien syihin, itse siirtymäprosessiin,
yritysten käyttöönottamiin uusiin käytäntöihin ja siirtymiseen
liittyviin sijoituksiin. Lisäksi yhdessä muista hieman erillisessä
koodiperheessä tutkin artikkeleiden kontekstia, eli esimerkiksi
maantieteellistä sijaintia sekä organisaation kokoa.

Ketteriin menetelmiin siirtymiseen johtaneet syyt voidaan jakaa
kolmeen pääkategoriaan. Ensimmäinen kategoria liittyy yritysten
liiketoimintaan, toinen prosesseihin ja kolmas projektien johtamiseen.

Tärkein liiketoiminnallinen syy siirtymälle oli vanhan prosessin
jäykkyys. Vesiputousmallin mukainen prosessi ei kyennyt vastaamaan
tehokkaasti asiakkaiden muuttuviin tarpeisiin ja vaatimuksiin. Toiseksi
vanhan prosessin ongelmana oli pitkä toimitusaika. Projektien
kesto oli usein jopa useita vuosia, minkä aikana sekä asiakkaan
liiketoimintaympäristö että markkinat saattoivat ehtiä muuttua
merkittävästi. Tämän vuoksi ohjelmistotuote, jonka rakentamiseen
käytettiin paljon aikaa ja voimavaroja, saattoi pahimmillaan ehtiä
vanhentua ennen julkaisuaan. Kolmas siirtymään johtanut syy oli
kilpailukyvyn ylläpitäminen. Nuoret ja pienikokoiset yritykset
pystyivät kilpailemaan samoista asiakkaista toimimalla nopeammin ja
joustavammin; ne pystyivät toimittamaan ohjelmistotuotteita asiakkaille
paremmin ketterien menetelmien avulla.

Vanhalla prosessilla oli myös ohjelmistotuotantoprosessiin
liittyviä ongelmia, jotka lisäsivät siirtymisinnokkuutta ketteriin
menetelmiin. Vesiputousmallia yritettiin parantaa uusilla versioilla
ja säännöillä, mikä kuitenkin loi nk. ``sääntöviidakon'' ja
teki prosessin jatkokehittämisestä käytännössä mahdotonta.
Prosessuaalisista ongelmista pyrittiin pääsemään eroon siirtymällä
ketteriin menetelmiin, joissa arvostettiin ihmisiä ja heidän
välistä kommunikaatiota sen sijaan, että tukeuduttaisiin jäykkiin
prosesseihin, malleihin ja työkaluihin.

Kolmas syy siirtymiseen liittyi projektin johtamiseen. Vanhassa
organisaatiomallissa yrityksen eri osastot toimivat toisistaan
eristyksissä niin kutsutuissa siiloissa. Niiden välinen viestintä
oli lähes olematonta, mikä teki projektien toimittamisesta
tehotonta. Eri siilot olivat eri aikoihin valmiita tiettyjen projektin
vaiheiden kanssa, jolloin osa siiloista joutui odottelemaan muiden
projektivaiheiden valmistumista ennen kuin pääsivät omien töidensä
kanssa eteenpäin. Ketterät menetelmät vastaavat tähän ongelmaan
uudenlaisella organisaatiorakenteella, jossa yrityksen sisäisiä
työryhmiä rakennetaan nimenomaan tiettyjen toiminnallisuuksien eikä
virkanimikkeiden perusteella.

Toiseksi pyrin tutkimuksessani selvittämään, miten
ohjelmistotuotantoprosessin muuttaminen etenee suurissa
organisaatioissa. Lähes kaikissa primäärilähteiden tutkimuksissa
muutos lähti ylimmän johdon aloitteesta näiden havaittua em{.}
liiketoimintaan liittyviä ongelmia. Ketteriä menetelmiä pidettiin
yleisesti potentiaalisena ratkaisuna. Valtaosassa tutkittuja yrityksiä
muutos oli kuitenkin hidas ja vaiheittainen. Ensin toteutettiin
taustatutkimuksia, joiden tarkoituksena oli selvittää työntekijöiden
senhetkistä tietotasoa ketteristä menetelmistä, yrityksen
suurimpia ongelmia senhetkisen prosessin kanssa, sekä sitä, mihin
ketteriin menetelmiin yrityksen kannattaisi siirtyä. Tutkimuksen ja
päätöksenteon jälkeen uusia menetelmiä lähdettiin kokeilemaan
ensin pilottiprojektien kautta. Pilotit kestivät puolesta vuodesta
vuoteen, jonka jälkeen uskallettiin siirtyä ketteriin menetelmiin koko
yrityksen tasolla.

Siirtymäaikana lähes kaikissa yrityksissä otettiin käyttöön
samoja käytäntöjä. Ensinnäkin yritykset hyödynsivät ahkerasti
ketteriin menetelmiin erikoistuneita ulkoisia konsultteja, joilla
oli kokemusta vastaavien muutosten toteuttamisesta muissa suurissa
organisaatioissa. Tämä koettiin tarpeelliseksi, koska suurin
epäonnistumisen uhka liittyi nimenomaan siihen, että työntekijöillä
ei mahdollisesti olisi tarpeeksi tietotaitoa. Tällöin he saattaisivat
palata vaistomaisesti takaisin vanhoihin tapoihin. Toisekseen valtaosa
yrityksistä otti käyttöön erilaisia tiedonjakoalustoja, joiden
tarkoituksena oli vastata osastojen siiloutumiseen liittyvään
ongelmaan ja lisätä toiminnan läpinäkyvyyttä koko yrityksen
tasolla. Kolmas käytäntö oli jatkuva integraatio. Tämä
tarkoitti sitä, että jokaisena päivänä tehdyt pienet muutokset
ohjelmistotuotteeseen vietiin välittömästi tuotantoon ja asiakkaan
testattavaksi. Tämä johti välittömään palautteeseen, jolloin
tuotetta voitiin kehittää yhteistyössä asiakkaan kanssa koko
ohjelmistokehitystyön ajan.

Johtopäätöksenä voidaan siis todeta, että vanhasta
vesiputousmallisesta ohjelmistokehityksestä siirrytään pois sen
vuoksi, että nykyaikana asiakkaiden ja markkinoiden toiveet ja tarpeet
muuttuvat nopeammin kuin mihin vanha prosessi pystyi vastaamaan.
Siirtyminen ketteriin menetelmiin toteutetaan pääsääntöisesti
ylimmän johdon aloitteesta hyvin suunnitellusti ja hallitusti. Lisäksi
siirtymän yhteydessä otetaan käyttöön uusia toimintatapoja, jotka
tehostavat yritysten toimintaa ja yhteistyötä sekä sisäisesti että
asiakkaan kanssa.
\end{document}
