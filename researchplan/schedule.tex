Tutkimustyön alkuvaihe tulee sisältämään ensisijaisesti
teoriapohjan luomista ja tiedonhakua. Erityisesti ensimmäisten
viikkojen aikana painopiste on huolellisessa suunnittelussa ja
tutkimuksen alustamisessa yhteistyössä ohjaajan kanssa. Ensimmäiseen
vedokseen mennessä tavoitteena on saada työn lopullinen rajaus
valmiiksi ja työn runko mahdollisimman lähelle lopullista tulosta.

Palautteen saamisen jälkeen lähden työstämään tutkimusta
palautteen mukaisesti. Todennäköisesti tässä vaiheessa suurin osa
työpanoksesta kohdistuu aineiston läpikäymiseen ja kirjallisuuden
analysoimiseen. Ennen toista vedosta tavoitteena on olla lähestulkoon
valmis tutkimuksen metodologian ja teoriapohjan kanssa.

Kolmanteen vedokseen mennessä työn painopiste on tulosten kokoamisessa
ja jäsentelyssä. Tämä luo pohjan tulosten tarkemmalle analyysille ja
itse johtopäätösten vetämiselle, jotka olisi tarkoitus olla valmiina
vedokseen IV mennessä.

Neljännen vedoksen palautteen jälkeen työn olisi tarkoitus olla
sisällön puolesta lähes valmis. Työpanos tässä vaiheessa
kohdistuisi enää pääsääntöisesti tutkimustulosten analyysin
viimeistelyyn. Kun tähän on päästy, työ vaatii enää yhteenvedon,
omaa pohdintaa tutkimuksen sujumisesta ja tuloksista, sekä omaa
kritiikkiä tehtyä tutkimusta ja sen tuloksia kohtaan.

Kun itse kirjallinen työ on tehty, alan työstämään esityskalvoja
seminaariesitelmää varten ja valmistelemaan työstä tiivistä,
mutta kattavaa esitystä. Tämän olisi tarkoitus olla valmis noin
viikkoa ennen seminaaripäivää, jotta aikaa jää vielä viime hetken
korjauksiin ja itse esityksen valmisteluun.
