Ketterä ohjelmistokehitys on noussut houkuttelevaksi vaihtoehdoksi
pienten ja keskisuurten organisaatioiden lisäksi myös suurille
organisaatioille. Ketterät metodit on suunniteltu alun perin
pienille ryhmille, mutta niitä on onnistuttu soveltavamaan myös
suurempien organisaatioiden tarpeisiin. Transformaatioon perinteisestä
vesiputousmallista ketteriin menetelmiin sisältyy kuitenkin haasteita,
ja transformaatio on jokaisella yrityksellä yksilöllinen. Suurissa
organisaatioissa haasteena on se, että ohjelmistokehitys ei yleensä
toimi itsenäisesti, vaan yhteistyössä muiden organisaatioiden
kanssa. Näitä voivat olla esimerkiksi design-, markkinointi- ja
henkilöstöhallinto-osastot. Koska tyypillisesti muut osastot eivät
toimi ketterien periaatteiden mukaan, ketterän ohjelmistokehityksen on
sovitettava toimintansa myös näiden sidosryhmien mukaan.

Tämän työn tarkoituksena on tarkastella minkälaisia haasteita
ja onnistumisia liittyy suurten organisaatioiden transformaatioihin
perinteisistä kehitysmalleista ketteriin metodeihin. Työ pohjautuu
Dikertin, Paasivaaran ja Lasseniuksen (2016) systemaattiseen
kirjallisuuskatsaukseen samasta aiheesta. Kyseinen tutkimus pohjautui
ennen vuotta 2014 suoritettuihin tutkimuksiin aiheesta. Tutkimuksessa
havaittiin, että varsinaisia transformaatiota koskevia tutkimuksia
ei oikeastaan juuri oltu suoritettu. Tutkimuksen lähdeaineistona
käytetyt aineistot olivat suurelta osin kokemuspohjaisia raportteja
ja ketteriä metodeita ylipäänsä koskevia tutkimuksia. Näin ollen
tässä tutkimuksessa perehdytään yksinomaan vuoden 2014 jälken
tehtyihin tutkimuksiin ja pyritään luomaan kattava katsaus uuteen
tutkimustietoon aiheesta.

