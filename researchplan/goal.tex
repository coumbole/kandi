Työn tarkoituksena on suorittaa katsaus tuoreeseen ilmiötä koskevaan
kirjallisuuteen. Koska tutkimuksen painopisteessä on nimenomaan
transformaatioiden aloittaminen ja eteneminen, tutkimuskysymykset
muodostetaan niiden pohjalta:

\begin{enumerate}
	\item Miksi suuret ohjelmistotuotanto-organisaatiot aloittavat
		transformaation ketteriin menetelmiin?
	\item Miten suuren mittakaavan ketterä transformaatioprosessi etenee?
\end{enumerate}

Koska tarkastelu on rajattu suuriin organisaatioihin, tutkittavien
organisaatioiden koko rajataan etukäteen vähintään 50 henkilöä tai
6 erillistä työryhmää sisältäviin organisaatioihin. Tämä rajaus
perustuu nimenomaan Dikertin, Paasivaaran ja Lasseniuksen aiempaan
tutkimukseen (2016). Tutkimuksessa tarkasteltujen esimerkkitapausten
perusteella tuota voitiin pitää riittävän suurena organisaationa
tarkasteltavan ongelman näkökulmasta. Nämä organisaatiot voivat
pitää sisällään myös henkilöitä, jotka eivät varsinaisesti
aktiivisesti osallistu ohjelmistokehitykseen, kuten esimerkiksi
ohjelmistoarkkitehtejä. Lisäksi tarkasteluun valitaan nimenomaan
sellaisia organisaatioita, jotka eivät ole aikaisemmin käyttäneet
ketteriä menetelmiä sovelluskehityksessä. Näin tutkimus rajataan
nimenomaan sellaisiin organisaatioihin, joilla transformaatio on ollut
vesiputousmallista suuren mittakaavan ketteriin menetelmiin. Pois siis
rajataan pelkästään ketteriä menetelmiä suureen mittakaavaan
skaalanneet yritykset.

Toisekseen tutkimus kohdistetaan nimenomaan sovelluskehityksen
kontekstissa sovellettaviin ketteriin menetelmiin. Näin rajataan pois
muussa yhteydessä, esimerkiksi teollisessa tuotannossa sovellettavat
ketterät menetelmät.

Tutkimuksen tavoitteena on kahden asetetun kysymyksen avulla luoda
kaksi näkökulmaa aiheeseen ja tehdä havaintoja, joita voidaan
hyödyntää tulevaisuudessa sekä tutkimuksessa että käytännössä.
Tutkimus suoritetaan kirjallisena analyysinä käyttäen olemassa olevaa
tutkimusaineistoa lähteinä. Tavoitteena on tuottaa käyttökelpoista
tietoa haasteellisena pidettävään suuren mittakaavan ketterän
kehityksen adoptoimiseen ja organisaation transformaatioon pois
perinteisistä vesiputousmalleista.
