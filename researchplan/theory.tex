Tutkimuksen tekeminen edellyttää laaja-alaista ymmärrystä
ohjelmistotuotannosta. On tärkeää ymmärtää ne eri
tuotannonvaiheet, vaatimukset ja sidosryhmät jotka vaikuttavat
kehitettävään ohjelmistoon. Lisäksi erityisen tärkeää on
syvällinen ymmärrys ohjelmistokehityksessä käytettävistä
malleista, sekä ohjelmiston elinkaaresta.

Koska tutkimus keskittyy erityisesti ketterien metodien adoptoimiseen
suurissa organisaatioissa, on luonnollisesti tärkeää ymmärtää,
mistä ketterässä ohjelmistokehityksessä on kyse. Tämän vuoksi
tutkimuksen aikana tulen perehtymään \textit{Agile Manifestiin}
ja eri ketterien menetelmien koulukuntien suosimiin malleihin.
Tämän lisäksi vaaditaan vahvaa ymmärrystä verrattavana olevasta
ohjelmistokehitysmallista – vesiputousmallista. Mallien tyypillisten
piirteiden ja ominaisuuksien ymmärtäminen on välttämätöntä, jotta
niiden välistä transformaatiota osataan arvioida kriittisesti ja
puolueettomasti.

Tutkimustyön näkökulmasta erityisen hyödyllisenä kurssina
opintojeni aikana pidän Software Engineering -kurssia. Kurssilla
luotiin kokonaisvaltainen käsitys ohjelmistotuotannon elinkaaren
jokaiseen vaiheeseen aina vaatimusten asettamisesta ylläpitoon.
Lisäksi kurssilla käsiteltiin myös kattavasti ohjelmistotuotannossa
aikaisemmin ja nykyään käytettäviä tuotantomalleja. Näiden
aihepiirien kurssimateriaalit tulevat palvelemaan myös tätä
tutkimusta teoriapohjan tukena.

Tutkimuksen tiedonhaussa hyödynnetään erilaisia alan tietokantoja
ja niistä löytyviä tieteellisiä tutkimuksia, raportteja ja
artikkeleita. Muutamien alustavien tietokantahakujen perusteella
tärkeimmiksi lähdetietokannoiksi tulevat todennäköisesti
muodostumaan IEEExplore ja Scopus. Hakutulosten määrä ja tulosten
otsikot vastasivat näissä tietokannoissa parhaiten tutkimuksen
tarpeita, kun hakuja suoritti muun muassa hakusanoilla ``agile transformation'' ja
``large-scale agile''.

Tutkimuksen tiedonhakuvaiheessa hakusanoja ja tietokantoja käytetään
laajemmin, ja aineisto tullaan karsimaan ensisijaisesti löydettyjen
artikkeleiden abstraktien perusteella. Työ tulee olemaan luonteltaan
kirjallisuustutkimus, jonka tarkoituksena on käsitellä aiheesta
jo tehtyjä tutkimuksia kriittisesti ja vetää johtopäätöksiä
teollisuudessa suoritetuista transformaatioista. Tavoitteena on
löytää näistä tyypillisimpiä haasteita ja onnistumisen avaimia ja
kehittää ymmärrystä siitä, mitä suurten organisaatioiden tulee
huomioida ketterien sovelluskehitysmetodien adoptoimisessa.
