None of the organizations I covered reported using established agile
scaling frameworks, such as Scaled Agile Framework (SAFe) [16], Large
Scale Scrum (LeSS) [17] or Disciplined Agile Framework (DA) [18]. The
articles did not state any partiular reasons why this is the case.
However, some papers were written by authors who had developed their own
frameworks [P2, P7], which explains their adoption method of choice.

Another possible reason is the relatively small size of some of the
organizations, despite being rated as large-scale in the context of this
study. In a SWD organization of 50–80 people the management and the
agile coaches and consultants they hired might have considered scaling
frameworks as overkill for their needs. Many of these scaling frameworks
support hundreds or even thousands of people, so in smaller companies
they may be able to tailor regular agile methods to their needs with the
assistance of agile coaches.
