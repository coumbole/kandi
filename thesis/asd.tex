Agile software development is a collection of methods describing an
alternative for traditional software developmet methods. The fundamental
idea is to build software iteratively in small increments while
constantly adapting the software according to customer feedback.
Feedback is acquired by delivering software to the customer already in
the early phases of development. With constant customer involvement
changes can be made rapidly even late in development. These principles were
explained in the ``agile manifesto'' (Beck et al 2001).

According to Mohammed and Abushama (2013), some of the most popular
agile methods in the industry are extreme programming and Scrum.
Scrum is a software development process for small teams that focuses
on project management. A scrum team develops a software product in
short sprints, which last from 1 to 4 weeks. In each sprint, the team
implements, tests, reviews and ships a finished piece of software
(Rising and Janoff 2000). A sprint is strictly timeboxed and the end
date typically does not change. If the team cannot finish what they
originally planned to during the sprint, the delivered functionality can
be reduced, but the length of the sprint always remains the same.

Extreme programming (XP) is a collection of practices which emphasize
on concurrent planning, implementing, testing and analyzing (Beck 1999). The
aim is to enable efficient incremental and iterative development
using e.g. continuous feedback from customer, test driven development
and continuous integration. Other important practices include pair
programming and constant refactoring (Beck 1999).
