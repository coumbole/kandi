All six case organizations applied at least some variant of Scrum in
their new software development processes. One organization reported
following mainly Scrum but complementing it with elements of Lean
principles in their software product development organization [P3].
One team in the same organization reported that they utilized mostly
Kanban, but combined it with some practices from Scrum, since especially
the retrospective was found essential to their way of working [P3].

The largest software organization studied in this paper (Samsung
Electronics) adopted Scrum of Scrums to coordinate the whole software
organization [P4]. Additionally, they applied practices such as
continuous deployment to support the Lean and Agile way of working.
However, inside the single Scrum teams they made use of single, separate
agile practices they found useful, such as pair-programming [P4].

Most organizations did not settle for using strictly Scrum. In five
out of six cases a combination of multiple agile methods was in
place. Nevertheless, all these cases included elements of Scrum. One
organization adopted a practice they labeled as ``Scrumban'', in which
they adopted the Kanban way of working, but included the concept of
sprint [P5]. Another example of using several methodologies side by
side was the large Telecom business [P6]. They adopted Scrum for their
information system and information technology development projects, but
XP for internal software development.

\bigskip
\centering
\captionof{table}{Contextual codes} \label{contextcodetable}
\begin{center}
    \begin{tabular}{l l }
        \bfseries{Contextual code} & \bfseries{Description} \\
        \hline
        Agile method & Agile methods that organizations started using \\
        Business area & Area of business in which the organization operates \\
        Geographical location & Where the organization is located \\
        Large scale definition & What the paper regarded as large scale software development \\
        Multisite / GSD & Mentioning of a multisite organization \\
        Organization size & Mentioning the size of the organization \\
        Research process & The paper defines a distinct research process \\
        \hline
    \end{tabular}
\end{center}
\justify

\bigskip
