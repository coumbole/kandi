
\begin{flushleft}
AALTO YLIOPISTO \hfill BACHELOR'S THESIS \\
PERUSTIETEIDEN KORKEAKOULU \hfill ABSTRACT \\
PL 11000 \\
00076 AALTO \\
\end{flushleft}

\begin{minipage}{15cm}
\begin{framed}
    Author: Ville Kumpulainen \\
    Title: Adopting agile software development in large organizations \\
    Program: Information Networks \\
    Date: 15.9.2017 \\
        Page count:~\pageref{LastPage} \\
        Responsible teacher: Miia Jaatinen \\
        Supervisor: Maria Paasivaara \\
        Language: English
\end{framed}
\end{minipage}

\bigskip

{\large\bfseries Abstract: \par} Agile methods have increased their
popularity among software development companies, but adopting them
in large organizations has not been and still is not easy nor
straightforward. I conducted a systematic literature review on why large
organizations adopt lean and agile software development methodologies
and how these transformations proceed. The keyword search found 1182
unique papers on the subject. Using a specific inclusion criteria,
6 of these papers were included in the final analysis. These papers
represented the highest quality regarding their informativeness
and narrative of the agile transformation process. The analysis of
these papers revealed four major categories of reasons for the agile
transformation: Business, process, management and organizational
reasons. The most significant reasons were excess process overhead and
lack of collaboration inside the organization. Most transformations
started from an executive management decision and proceeded in a
planned, step-wise manner.

\medskip

\begin{minipage}{15cm}
\begin{framed}
        Keywords: Agile software development, large-scale agile, organizational
        transformation, adopting agile software development
\end{framed}
\end{minipage}
