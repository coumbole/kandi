One significant limitation in this literature review is the limited
quantity of primary studies. The abstract based filtering excluded a
significant amount of papers, and the full text filtering some more,
leaving only 9 papers left for the coding and analysis. This is a
relatively low sample size. However, it it does indicate that the focus
in the field of large-scale agile research has shifted from case studies
and other empirical research methods to more theoretical approaches.
In the study selection phase I noticed that a large number of studies
focused on modeling transformations and creating metrics to analyze the
performance during and after agile transformations.

Another limitation I faced in the research process was the keyword-based
search. It is possible that some potential sources were missed due to
not falling under the search queries. The inclusion criteria are quite
strict, and included papers have to cover all four aspects described.

Lastly the quality of the primary sources is also a concern to address.
In general the papers did not have a clearly defined research method or
research questions they were attempting to answer. Instead they were
experience reports written usually by members of the organization.
Due to the researchers not being outsiders, they may be subject to
showing their organizations in a more positive light than what the
situation really was. None of the organizations reported any major
struggle or failures during the transformation, which seems at least
slightly questionable, since large-scale agile transformations usually
involve some obstacles.
