As part of the transformation process, some new practices were
introduced to organizations' software development processes. Coaching
was among the most common practices to be utilized during the
transformation. [P3, P4, P5] However, coaching is most valuable during
the transformation when the developers are new to agile methods and
principles. Other practices that would carry on being made use of
include continuous integration, communities of practice and different
knowledge sharing practices.

{\bfseries Continuous integration}. According to Fowler [9],
\textit{``Continuous Integration is a software development practice
where members of a team integrate their work frequently, usually each
person integrates at least daily – leading to multiple integrations
per day. Each integration is verified by an automated build (including
test) to detect integration errors as quickly as possible''}. This
approach was adopted at Ericsson as well to increase better means to
support faster feedback with the R\&D organization. In practice, they
built the automated build and test machinery and started co-operating
closely with feature development and system integration teams, resulting
in faster feature delivery and more streamlined operations. [P3]

{\bfseries Communities of Practice}. Another mechanism to increase
communication and knowledge sharing was the so-called community of
practice [P5]. These helped teams assess pressing issues with collagues
efficiently and increased motivation among employees. A business analyst
at the organization commented about the importance of communities of
practice: \textit{``[Equally] important as training people is sharing
what we are learning with one another, and here is a good way to do it.
I hope they do not cut this practice off''}. [P5]

{\bfseries Knowledge sharing practices}. Ericsson introduced a
multitude of new knowledge sharing practices in order to increase
inter-organizational communication and convert tacit knowledge to a
written format. Duka reported that they introduced a practice called
``Current best thinking'' to emphasize the short term planning and to
encourage everyone to question and look for alternatives to current
processes. Another practice they introduced was an open question session
called ``Fast Forward Friday'', in which everyone at the organization
could ask and get answers from the management on the current state of
things. These sessions were found valuable by both the employees and the
management. [P3]
