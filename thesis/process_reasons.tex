{\bfseries Late integration, testing and feedback}. The nature of
waterfall development made the testing and feedback cycles very long.
Anwar et al considered reducing the total testing effort one of the most
significant reasons for improvement. Especially late user acceptance
testing had caused major budget overruns in several projects. [P1]

{\bfseries Old process does not scale up}. Constant improvements and
process tailoring had helped to create a relatively well working and
mature development model in some cases [P1]. However, while the old
process was working well for old customers, it did not work as well with
new projects. Wells et al had reported the heavy emphasis on upfront
estimation and planning to make the waterfall model monolithic and
heavy, leading to inefficiencies. Having to go all the way back to the
beginning of the process to introduce changes was not feasible in most
projects. [P6]

{\bfseries Process overhead}. Heavy and well-defined process works well
in system or safety critical software development but not as well in
novel or smaller scale projects. Excessive process overhead had caused software
companies to ``over-architect rather than doing the job''. [P6] Moreover,
releaving the rigidity in plan-driven development was a major reason to initiate
the agile transformation at Samsung electronics as well [P4].
