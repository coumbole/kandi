Hui [P7] introduced a new way to lead organizations through agile
transformation called Lean Change. The fundamental idea is to run agile
transformations in a lean manner like startups as described by Ries in
The Lean Startup Methodology [13]. The major idea is to first assess the
situation, then start with a very basic, for example Kanban, workflow
and then introduce changes and adaptions to the process as Minimum
Viable Changes (MVC). These small adjustments can be validated quickly
and incorporated to the workflow if they work and leave out if they do
not. This makes introducing modifications to the process very lean and
lightweight. [P7]

Hui provided quite detailed descriptions on how they implemented several
MVCs in order to improve their development processes. As a starting
point to their agile transformation, they first implemented a so-called
`as-is-Kanban'' to gather information of the current state of projects.
Having found that useful they decided to keep this method and develop it
further with the next MVC\@. This was called `Enterprise Kanban'', and
its purpose was to help visualize the entire IT portfolio of the company
in order to help with issues related to multitasking and inter-team
communication and collaboration. [P7]
