The searches were performed on four different online databases as
described in table \ref{databasetable}\@. This study only focuses on most recent studies to limit
the amount of data, as the scope of this study is limited. Therefore, only
studies from 2013 onwards have been included in the searches.

\begin{center}
\captionof{table}{Databases}\label{databasetable}
\begin{tabular}{l l c }
    \bfseries{Database} & \bfseries{URL} & \bfseries{\# of matches} \\
    \hline
    IEEExplore & http://ieeexplore.ieee.org & 471 \\
    ACM & http://dl.acm.org & 153 \\
    Scopus & http://www.scopus.com/home.url & 844 \\
	Web of Knowledge & http://apps.webofknowledge.com &  615 \\
\end{tabular}
\end{center}

\vspace{1cm}

The search strings were constructed using boolean operators and aspects
presented section 3.2.1, as demonstrated in table \ref{keywordtable}\@. All databases
used in this study supported complex boolean-based search strings,
which greatly increased the accuracy of the searches. Preliminary test
searches with trivial keywords (such as ``agile software development'' and
``large scale agile'') proved that among the interesting articles there is
also a vast amount of uninteresting papers. Being able to filter some
of those at the search phase reduced the amount of manual labour in the
next step. However, the preliminary searches also helped to identify
that some actually interesting papers were not left out of the results
by accident due to too complex search string.

\medskip
\centering
\captionof{table}{Aspects and related search terms} \label{keywordtable}
\begin{tabular}{l p{30em}}
    \bfseries{Aspect} & \bfseries{Keywords} \\
    \hline
    Agile methods & agile, scrum,``extreme programmin'', waterfall,``plan–drive'', RUP \\
    Organizational transformation & transform*, transiti*, migrat*, journey, adopt*, deploy, introduc*,``roll–ou'', rollout \\
    Only software related articles & (software OR (conference=``agile, xp, icgse, ics'')) AND NOT (title+abs=``manufacturin'' OR conference=``agile manufacturin'') \\
    \hline
\end{tabular}

\justify

\medskip
