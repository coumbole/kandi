Agile software development methods were originally developed for small
software development teams. However, their proved success has made it an
attractive alternative for larger organizations as well. Nevertheless,
it has been noticed that adopting agile methods in large organizations
is more difficult. [10] One significant factor
that makes the adoption more challenging is that in small organizations
software development teams are typically more independent. Large
software teams have more dependencies between other organizations
and departmens, increasing the need for formal documentation and
communication [11]. Agile methods are largely based
on quick, informal communication between team members, such as daily
scrums. Once organization and team sizes grow large enough, informal
communication is not sufficient to keep all stakeholders up to date with
the development progress [4, 11]

Agile software development also causes issues with business processes.
In traditional software development the phases of the project are easy
to define. This makes setting up milestones, measuring progress and
writing contracts easier, as everything can be decided beforehand. With
agile development, tracking progress can be more challenging since the
requirements and planned features might change throughout the project.
This also makes pricing and effort estimation more difficult, which
makes traditional software development processes more attractive from
management point of view [3].
