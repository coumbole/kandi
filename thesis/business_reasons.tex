One of the most typical reasons for initiating an agile transformation
had to do with the business case.

{\bfseries Accomodating change needed}. A major issue for
several organizations was that they were unable to respond to change
requests quickly enough [P2, P3, P5]. One commented this issue reported: \textit{it became clear
that we needed to do a proactive change in order to more flexibly react
to customer wishes}. [P3]

{\bfseries Demand for faster delivery}. Several companies also reported
pressure to deliver working software faster [P1, P2, P3, P6]. The
pressure was especially strong in companies building primarily custom
software for external customers. Smaller companies that had implemented
agile methods to their processes were fulfilling customers' needs
better. [P6] Wells et al stated: \textit{For this technology{-}intensive
company the challenge of being able to compete in speed to market was
achieved through the creation of a culture and mind-set ready to respond
rapidly to change, external needs and technological developments.}

{\bfseries Need to increase innovation}. One organization also reported
the motivation for agile transformation to come from the need to work
more innovatively: \textit{We seek agility as a ``driving force'' to
innovation}. [P5].

{\bfseries Remain competitive}. This was one of the most common reasons for
agile transformations. Organizations needed to undergo fundamental changes to
their software development processes in order to achive the improvements they
were looking for. Constant process improvements for their waterfall process had
made further improving difficult, as reported by Anwar et al: \textit{This
accumulation of tailoring rules has caused what we call ``overall process
corrosion'' – a process that works but is very hard to evolve}. [P1]

{\bfseries Reducing time to market}. Long iterations and lead times have become
a major issue for companies delivering customized software to external
customers. The long time window between the customer requesting a piece of
software and receiving it was concidered a major motivation for the
transformation. [P5]
