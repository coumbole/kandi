In addition to providing training, all companies invested in the
transformation in other ways as well.

{\bfseries Preliminary research}. Anwar at al reported the company
to have been cautious in avoiding previous improvement pitfalls,
and conducting a research of the applicability of the agile process
before starting the transformation [P1]. Their research question was
\textit{``Can agile process improve the organization's performance
indicators while maintining the process maturity level?}''. In order to
determine the answer, they conducted an action research and as a result
decided to initiate the transformation. [P1]

Ayed et al also reported studying the organization's readiness to adopt
agile methods before the transformation. They conducted an interview
based study to all relevant stakeholders from different business
units to determine their current development process, motivation and
reluctance towards adoopting agile software development process. Using
this data, they performed a SWOT risk analysis to prepare for possible
pitfalls that might occur during or after the transformation. [P2]

{\bfseries Consultants}. Two organizations hired consultants to support
the transformation process. As stated stated in section 4.3.3, one
company had hired two process improvement engineers for the sole
purpose of going through the transformation [P1]. Roman et al reported
hiring external consultants as agile coaches to closely support the
transformation: \textit{``Local coaches were hired to support each of
the IT offices. We looked for experienced professionals who have faced
similar issues [as] ours in other large corporations''}. [P5]
