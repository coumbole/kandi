As this paper provided recent insight to the subject already covered by
Dikert et al [12], this is a prime opportunity to compare findings between a
four-year-old study and the recent one.


{\bfseries Differences in adoption motives}. Dikert et al found the
most prominent reason for initiating an agile transformation to be the
demand for faster delivery. The second most often cited reason was
the need to operate more flexibly and respond to change better. [12]
However, according to the studies covered in this paper the most urgent
motive behind the transformations was the lack of collaboration due
to organizational silos, followed by the need to accomodate to change
better and demand for faster delivery. This indicates that the need for
between-team communication inside software development organizations
has increased in importance in the recent years. The reason for this is
not completely clear and could serve as a starting point for further
research. Still, the three next prominent reasons all had to do with
business reasons as with the previous study, which indicates that the
underlying problem behind waterfall software development still causes
trouble in companies.


{\bfseries Differences in transformation processes}. As for how the
transformations proceed, my findings were very similar to those of
Dikert et al's. Almost all transformations I covered in this paper were
initiated by top management and utilized a step-wise approach. Dikert
et al described typical transformations to have used pilot projects,
hired external consultants and coaches and invested in the software
development staff's training in agile methods. All these characteristics
applied to the majority of studies I covered.
