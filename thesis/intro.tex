Agile software development has become an appealing alternative for
large organizations striving to improve their performance [19].
Even though agile methods were originally designed for small teams,
their benefits have made them attractive for larger organizations as
well [20]. The transformation from traditional waterfall model to
agile methods in large organizations is challenging, as there is no
universal framework that could be directly applied with guaranteed
results. One significant challenge is that agile software development
methods originally concerned only software development teams. However,
in large organizations software development is rarely conducted in
isolation. On the contrary, it is done in cooperation with for
example design, marketing and human resources departments. Since these
organizations have their own processes that likely do not follow
agile methods, large scale agile transformation must take into account
all stakeholders' needs.

The purpose of this paper is to observe why large software
development organizations initiate agile transformation and how these
transformations proceed. The paper is partly based on Dikert et al's
paper \textit{Challenges and success factors for large-scale agile
transformations: A systematic literature review} [5]. This study was
based on articles, experience reports and other papers published in 2013
and earlier. They found that there are not many studies regarding large
scale agile transformations, and most of their referenced articles were
experience reports. Thus, this research studies articles and papers
published in 2013 and after that and attempts to create a comprehensive
review of new research data regarding the subject.
