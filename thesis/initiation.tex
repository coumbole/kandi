Most studies reported the transformation to have begun from a management
decision. The increased interest in a new and more efficient software
development process rose usually from business or process management
issues as discovered in section 4.2. Therefore, in most cases the
software development team typically had more or less neutral stance on
the current process.

In order to promote the transformation efficiently, some companies
established dedicated organizations for driving the process. For
example, one company created a so-called ``Software Process Improvement
Group'' whose dedicated mission was to align the new agile development
process with the new strategic objectives set by senior management [P1].
Another example of this occured at Samsung Electronics, where they
established an ``Agile Office'' to promote agile methods and decrease
rigidity organization wide. [P4]

In some cases the transformation was driven heavily by a single person.
In two companies a newly hired manager had previously worked with
agile methodologies and started to push heavily towards it in the new
environment, resulting in initiating the agile transformation [P1, P5].

In one organization the transformation actually begun from the bottom
of the organization. According to Ayed at al, their desire to adopt an
agile approach came from a single development team in the whole software
development organization. The team was seeking to improve their own work
and they introduced Scrum in their own team. Subsequently the whole
organization adopted Scrum, as it had resulted in increased productivity
in the single team. [P2].

As with the aforementioned case, most of the other transformations
proceeded in a step-wise manner. Typically, when a team proved the new
methodology to be efficient, the natural desire to spread agile values
among other teams followed [P2]. In top-down mandated transformations
the management typically was a bit cautious at first, and started small
with the transformations [P3]. For example, at Samsung Electronics the
Agile Office laid down a step-by-step plan to expand the agile culture
gradually throughout the organization [P4].

Only one organization started a ``big-bang'' transformation to agile
methods. The new CIO announced a world-wide transition to begin in the
company to move away from traditional software development methods.
However, there was still some preliminary research and planning in
place even if the transition itself happened in a rapid fashion. As
the organization was large and operating on 5 continents, even a
``big-bang'' transition took over a year to complete. [P5]
