Introducing a whole new software development process to an organization
requires someone to lead the transformation. Merely announcing a new
strategy on its own will not suffice. As such, further investments,
training and change leading is required. Each of the primary studies
presented their own approaches to leading the agile transformation.

Anwar et al reported that their company, after establishing the Software
Process Improvement Group, hired two full-time process improvement
engineers to lead the transformation. In addition, they had hired a new
software delivery manager that was also pushing for going agile in a
lower level. [P1]

At Samsung Electronics the change leadership was appointed to the Agile
Office [P4], whose primary mission was to spread knowledge about agile
methodologies and organize cross-functional teams to reduce the rigidity
of the organization. Therefore, while the initiative itself came from
top management, the change actually took place at a lower level. The
agile office organized trainings for all different roles in a Scrum team
and supplied all software teams in the company with coaches. [P4]
