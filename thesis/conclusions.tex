This paper makes a contribution to recent research regarding large-scale
agile transformations. While research has covered extensively the
usage agile methods in small, medium and large organizations, the
transformations in especially large organizations has not been
covered very broadly. This paper compiles reported large-scale agile
transformations from past four years creating a summary of recent
research data to be used by practitioners and researchers alike.

The first goal of this paper was to find the most typical reasons behind
large organizations undertaking agile transformations. The secondary
goal was to cover what measures these organizations take in order to
succeed in this transformation in which most companies struggle.

The main reason for moving away from waterfall development to agile
methods was to increase collaboration inside the organization, followed
by the need to respond better to changing markets and change requests.

Most transformations proceeded in a planned, step-wise manner. They
began with a research and training, followed by a pilot project or
two and finally resulting in company-wide roll-out to agile methods.
Furthermore, the transformations were initiated by top management as
a response to dissatisfaction with the current process and state of
business.

– Elaborate previous points a bit more


- Point out some further research topics
