The nine primary sources were coded using the Atlas.ti qualitative data
analysis software. The coding process followed similar principles as
Dikert's, Paasivaara's and Lassenius' systematic literary review of the
challenges and success factors for large-scale agile transformations
[5]. Thus, five of the seven code families used in the aforementioned
study were used in the coding of these studies. The two left out
families, \textit{challenges} and \textit{success factors} were not
considered relevant for this study, since the reserch questions and
goals are different. A description of the code families as well as
examples ofo certain codes are show in Table 4. The table also presents
the total number of codes and quotations created. A single quotation may
contain several codes and belong to multiple code families, which is why
the total number of quotation is less than the sum of quotations in each
family.

\bigskip
\centering

\captionof{table}{Code families, codes and quotations} \label{codefamilytable}
\begin{center}
    \begin{tabular}{l p{10em} p{11em} c c }
        \bfseries{Code family} & \bfseries{Description} & \bfseries{Examples} & \bfseries{Codes} & \bfseries{Quotations}\\
        \hline
        RQ1: Reasons to change & Reasons to start the transformation & reducing time-to-market & 12& 45 \\
        RQ2: Transformation process & Statements describing the transformation process & top-down, big bang, step-wise & 4 & 19 \\
        Practices & Practices used or established during the transformation & piloting, coaching, continuous integration & 5 & 17 \\
        Investing in change & Factors presenting how the organization is investing in the transformation & training, consultants, tools & 4 & 15 \\
        Contextual & Contextual codes defined in table~\ref{contextcodetable} & agile method, organization size, large-scale definition & 7 & 42 \\
        \hline
    \end{tabular}
\end{center}
\justify

\bigskip
